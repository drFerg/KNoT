\section{What is the ``Internet of Things''} % (fold)
\label{sec:internet_of_things_paradigm}
This section describes the concept of the ``Internet of Things'', its slow development up to the present, and the motivations as to why one would want to create an ``Internet of Things''.

\subsection{Past, Present and Future} % (fold)
\label{sub:past_present_and_future}

% subsection past_present_and_future (end)
The standard model of how we use computers and the Internet today revolves around the idea of an interconnected network of servers, routers and data centres around which users connect using their personal computers to access, input, manipulate and retrieve data. 

This model heavily relies upon the users to provide the data by which the Internet is powered. Without this data the Internet would be a barren place, with nothing to search for, sell, buy, share, watch, listen to, play, analyse or learn from.
Users from all around the world have contributed to make the Internet \textit{the} single biggest resource in the world by snapping photos, capturing videos, writing blogs, creating websites, commenting, discussing, reviewing, buying, selling and inputting data. So much data in fact that the estimated size of the Internet in 2009 was 500 exabytes (that's 500 BILLION gigabytes), of which 70\% was contributed to by users.\cite{Size}   

% INSERT DIAGRAM HERE
Within this model there exists two significant problems posed by users; time and accuracy. In terms of time, users only have so many hours in a day to input data; which can only be so accurate, as all users are prone to error through one means or another.
These two problems make it difficult to observe our world and represent it in an accurate and reliable digital form.

The idea of taking this responsibility of inputting data away from the user and giving it to the machine is where the ``Internet of Things'' term initially came from. The term itself was thought to have first been coined by Kevin Ashton\cite{K.Ashton} in 1999, who at the time was interested in linking up a company's supply chain to the Internet using RFID technology to allow for autonomously monitoring its state.
Sadly, at that time the idea progressed little and didn't garner much support.

Fast forward to 2013, the ``Year of the Internet of Things'' as declared by the MIT Technology Review\cite{2013IoT} and observed in the Consumer Electronics Show 2013 (CES) with manufacturers introducing everything from smart fridges to smart weighing scales. Concurrently, low power devices have become increasingly affordable and small enough to embed into other devices, making it much more possible to create a truly autonomous system, either in the home, office or car, wherein the user doesn't have to interact directly with any of the devices but instead just go about daily life, which, in the background, is enhanced by these ubiquitous devices operating together.

With all the possible ``Things'' that can exist, the core functionality in the ``Internet of Things'' can be simply modelled as a closed loop system in which a ``Thing'' can sense, process and interact with the real world as well as other devices and services. This system is represented at its simplest by an if statement, \textit{if \textbf{event occurs} then \textbf{do action}}, which provides developers and users alike an easy to understand yet powerful model to create rules for autonomous interaction.i.e.,
\begin{center}
	\textit{if \textbf{temp \textless  20C} then \textbf{Turn ON heating}}

	\textit{if \textbf{washing machine done} then \textbf{Text John}}
\end{center}

There currently exists a variety of online services and devices which give the end-user this ability to do such tasks, such as the Twine device \cite{Twine} which is a small low power device with a variety of sensors and a WI-FI radio. The goal of the device was to make it simple and easy to set up and use; the user just puts the device in the desired environment, perhaps on a washing machine, sets up some rules (``When accelerometer is knocked Then tweet''), and when the condition is true, the preset action, in this case tweeting, is performed. Whilst this gives the user great flexibility to perform actions automatically, the limitations show when trying to interact with more than one Twine or with other devices and not just Internet services. 

\begin{figure}[h]
	\centering
	\begin{minipage}{.5\textwidth}
		\includegraphics[width=.6\linewidth]{images/twine-device.jpg}
		\captionof{figure}{Twine IoT device.}
		\label{fig:twine_device}
	\end{minipage}%
	\begin{minipage}{.5\textwidth}
		\includegraphics[width=.6\linewidth]{images/twine-rules.jpg}
		\captionof{figure}{Twine rule, When x Then y}
	\end{minipage}
\end{figure}

The Twine device isn't the only device out there, Smart Things\cite{SmartThings} goes one step further than Twine and gives the user a network of devices connected to the Internet through a central hub which the user can then interact with online.

The main problem that exists and will continue to do so is that the digital environment is filled with a huge variety of heterogeneous devices; the key to the future of the ``Internet of Things'' is to find a way to abstract away the differences between these devices and create a simple platform on which all types of devices can communicate to build a more rich and powerful ``Internet of Things''.
\newpage
\subsection{Motivation} % (fold)
\label{sub:motivation}
Like any new paradigm of computing, there must be sufficient motivation and interest for it to be picked up, used and developed further. In the case of the ``Internet of Things'', there are many cases why such a paradigm is of great interest, both to consumers and organisations.

\begin{itemize}
	\item Workload 
	\begin{itemize}
		\item As mentioned previously, the core principle of the ``Internet of Things'' is to stop the user from having to consciously input data into systems, instead, replacing them with new devices(sensors, actuators). By doing so, the user's workload is reduced, giving them more time to do other activities, either of more interest or importance. 
	\end{itemize}
	\item Cost and Eco-savings
	\begin{itemize}
		\item One of the key benefits from collecting data from these devices rather than users is that more accurate and reliable data can be gathered, from which smarter decisions can be made for the user, often resulting in energy savings i.e., turning off unnecessary lights or optimising heating schedules. These energy savings not only benefit the user in terms of monetary cost but also benefit the planet, making it valuable to both the individual and larger community as a whole.
	\end{itemize}
	\item Security
	\begin{itemize}
		\item By utilising these devices, not only can data be collected reliably but actions can also be executed reliably, unlike users. This can benefit the user in terms of security, ensuring that an important action is always carried out i.e., automatic door and windows locking upon exiting the house.
	\end{itemize}
	\item Remote
	\begin{itemize}
		\item Because of the connectivity of these devices, users can interact with them from anywhere with an Internet connection, giving them access to change, update and ensure the system is working correctly i.e., check oven is turned off and front door locked when out.
	\end{itemize}
\end{itemize}


% Reducing work load giving people back more time, simplifying life
% Improving security, 
% ECO
% COST SAVING
% Improving living standards

% subsection motivation (end)

\section{Typical ``Internet of Things'' examples} % (fold)
\label{sec:typical_}
In order to better understand the possibilities and typical scenarios surrounding an ``Internet of Things'' equipped home or office, this section will outline two ideal examples of an ``Internet of Things'' network embedded in the home and office, discuss their respective benefits as well as the current state of ``Internet of Things'' networks.

\subsection{``Internet of Things'' at Home} % (fold)
\label{sub:home}
The most common association with the ``Internet of Things'' is the ``Home of the Future'', which gives the consumer the impression that the static home they are used to will come alive with technology, to make life at home far simpler and more enjoyable. Therefore, this initial example will focus on the ``Internet of Things'' around the home and demonstrate several uses of it.

The primary goal of an ``Internet of Things'' network embedded in the home is to create an awareness of the user. By making the home aware of the user, it can observe them in real-time and determine \textit{smart} things to do autonomously within the home. These could relate to security, such as locking and unlocking doors and windows when the user enters or leaves, or relate to convenience, such as turning on and off appropriate lights as the user moves between rooms, turning on the heating in anticipation of the user's arrival. 
Over longer periods of time it could keep track of the user's habits and adapt heating schedules or open the blinds when the user usually goes to sleep or wakes, it could even turn on the coffee pot so when the user awakes in the morning a fresh cup of coffee awaits them.

Other examples are less concerned with the user but instead with the safety and upkeep of the home, by embedding sensors within objects in the home autonomous maintenance can be carried out. Some examples of this could be, automatic plant watering when low soil moisture is detected, or autonomous vacuuming at certain periods of time when the user is away.\cite{Roomba} By creating these possibilities, it not only reduces the workload for the user but it also allows the user to do things which matter most to them, and of course it stops plants from being forgotten about.

\subsection{``Internet of Things'' at Work} % (fold)
\label{sub:work}
Much of the benefits from deploying an ``Internet of Things'' at home also apply when deployed in a work environment, such as energy savings, reducing workload and increasing security.

In the workplace, the concept of the ``Internet of Things'' being aware of the user could be taken advantage of to a much greater extent to increase productivity, efficiency and independence. This could be possible by allowing workers to be automatically checked in and out of their office and other key locations within the workplace, such as the cafeteria or meeting rooms, with varying granularity based on privacy. By doing so, not only can a worker forget about informing colleagues of their location through punch cards or switchboards, but it can also help keep their colleagues to be more informed about each other, so that instead of finding an empty office after walking from the other side of the building, time can be better spent.

% subsection  (end)
%  home (end)


% subsection typical_ (end)
 % Kevin Ashton, RFID, coined IoT
 % http://www.rfidjournal.com/article/view/4986
 % Giving everyday objects a digital presence and creating machine to machine connections
 % Autonomy, semi-autonomy
 % Taking the human out of the equation and empowering the computer to produce and consume its own data to make our world more 
 % efficient, less wasteful, more time and less energy.
 % Slow start, not much going on.

 % 2012/13 jump start, lots of manufacturers
 % open hardware initiatives
 % linking home automation, security and IoT i.e. toaster/coffee machine
 % 

 %%%%%%%%%%%%%%%%%%%%%%%%%%%%%%%%%%%%%%%%%%%%%%
% HOME AUTOMATION? ??????????????????????
%%%%%%%%%%%%%%%%%%%%%%%%%%%%%%%%%%%%%%%%%%%%%%%%%%


% section internet_of_things_paradigm (end)
\newpage
\section{Open source constrained devices} % (fold)
\label{sec:open_source_constrained_devices}

In the past 5-10 years, there have been many attempts at creating an affordable, low power and approachable electronics platform such as Arduino, Netduino, BeagleBone, Teensy and MSP430 Launchpad. All of which have taken very different approaches, some opting for absolute low cost (MSP430 Launchpad), whilst others aimed to be fully featured and powerful devices (BeagleBone). Other devices such as the TelosB mote, whilst not cheap, has fuelled academic research into new ways of designing and implementing Wireless Sensor Networks. More recently, the Raspberry Pi has created a whole new market of super, low-cost, yet moderately-powerful computers aimed at education and hobbyists.

In the rest of this section a variety of different devices and platforms will be discussed including the Arduino, TelosB mote, Raspberry Pi and other ``Internet of Things'' related devices. 

\subsection{Arduino} % (fold)
\label{sub:arduino}

The Arduino was born in 2005 at an Italian university, Interaction Design Institute Ivrea, out of the necessity of creating a cheap and approachable electronics platform which could enable design students to create interaction design projects without the need of an electronics background.
The main device which was created and has remained much the same since, is based on an Amtel ATmega328 8bit micro-controller running at 16MHz with 2KB of RAM and 32KB of storage for programmes written in a variation of C/C++. The board itself maps out the micro-controller's mix of 20 digital and analogue input/output pins and supports several standardised protocols for communicating with other devices such as I\textsuperscript{2}C\footnote{I\textsuperscript{2}C - Inter-Integrated Circuit} and UART\footnote{UART - Universal Asynchronous Reciever/Transmitter}. 

But the key to the Arduino Platform is not the micro-controller itself but instead the design, software and support provided by the creators and other developers. The other major factor to its success is that the device, along with the documentation and support, are all open source, thus allowing anyone to learn from, replicate and expand upon the platform in any way they wish.

An example of how these factors have had a hugely positive effect is something which Arduino calls ``Shields''. These shields plug in on top of the Arduino board and contain standard components which can add additional features such as WI-FI, Ethernet, sound, motor control etc. Rather than users being required to find, purchase and solder the required components to add such features, these pre-built shields provide it in simple and readily available package, made by a variety of manufacturers.

Since its creation, the Arduino platform has created a range of Arduino named devices and shields resulting in a following of over 300,000 users\cite{ArduinoNumbers} and support from many manufacturers and distributors worldwide. 

Due to Arduino's open source licensing policy, many new start-ups have been able to quickly design prototypes and products using the Arduino platform which have been taken to market in various refined forms. Often the same micro-controller which powers the Arduino is kept and the board miniaturised to fit the product's needs. Products such as the Internet ``Thing'' Twine\cite{Twine} and the Smart Things ``Internet of Things'' eco-system have taken this approach\cite{SmartThings}.

However, whilst there is an abundance of Arduino devices in the wild with many being used as an Internet ``Thing'', there is yet to be created an open and compatible method to easily network a group of such devices together to form a connected ``Internet of Things'' network.

\begin{comment}
OPEN SOURCE
Micro-controller, cheap, easy, PIC chips were difficult, lowered barrier of entry, provided large support and huge following
Italian made, several iterations on size and power
Good starting point for development as there are many in existence
Well adopted
Based on C/C++ with a few tweaks, offers shields for expandability i.e. Ethernet, WI-FI
very good at sensing and doing things
16MHz
no threading
\end{comment}
% subsection Arduino (end)

\newpage
\subsection{Raspberry Pi} % (fold)
\label{sub:raspberry_pi}
Launched in 2012 the Raspberry Pi, a \$35 credit card sized computer, was eagerly anticipated to change the computing landscape. The charity behind it, the Raspberry Pi foundation, had one main goal; to refresh and promote the teaching of computer science in schools.

Similar to the Arduino Platform, most of the hardware and software for the device is open source with the aim of letting adults and children alike get their hands dirty learning about computing without the worry of breaking an expensive computer.

The Raspberry Pi itself is a moderately well powered, single-board computer capable of running a wide variety of Linux-based operating systems. It's host to a 700MHz ARM CPU, 512MB RAM and a variety of inputs/outputs including an Ethernet port. These features, along with some additional GPIO\footnote{General Purpose Input Output} pins, just like the Arduino, allow it to sense and control the world around it, thus making it a perfect ``Thing''.

Because of the extremely low price point for such hardware, the Raspberry Pi was a huge success selling, over 500,000 units in the first 8 months, only limited by their production speed.\cite{RaspberryPiSold}

Even before it's release, the community support and ideas were non-stop, everything from home-media centres\cite{Raspbmc} to Lego super computers\footnote{Even within the School of Computing Science, UoG}\cite{LegoSuperComputer} were designed and created from Raspberry Pis.

Whilst the support for utilising the GPIO pins is not yet as well developed as the Arduino Platform, the power of the device combined with its low price point and connectivity means that it can be very easily incorporated into a ``Internet of Things'' network with little additional hardware or cost.

Another considerable point for developing on the Raspberry Pi is that because of its ability to run Linux, a much larger variety of programming languages can be used, including high level languages like Python, C++ and Java.
% ADD PROGRAMMING LANGUAGE PART

\begin{comment}
new, more powerful open device. runs a real operating system with gpio options
perhaps upper limit of power for this protocol
most languages possible, threading etc
\end{comment}
% subsection raspberry_pi (end)

\subsection{TelosB Motes} % (fold)
\label{sub:telos_b_motes}

With a significant uptake of devices in academia, TelosB motes are one of the go to platforms for wireless sensor networks research. Whilst they may fit under a different paradigm, the IoT can be seen as a specialisation of a wireless sensor network. Although the are extremely constrained, they are a good, low-end benchmark for which to test and develop software for use on constrained devices. 

The device was developed at the University of California, Berkeley, and spun off into a separate company. As a result, these devices have fuelled many academic studies and courses in relation to developing low power wireless sensor networks and researching applications for them. With very limited resources, 8MHz CPU, 10KB RAM, 48KB ROM, light/temperature sensors, battery powered and a low-power 802.15.4 (250 kbps)radio, these devices require new ways of thinking in order to design, program and deploy them. Many scenarios must also take into consideration the environment in which they will be deployed, which are often hostile and unpredictable that can cause nodes to disappear from the network sporadically. Such environments have ranged from forests for fire detection\cite{FireDetection} to monitoring livestock \cite{Livestock}.

A significant part of the development for these types of devices has surrounded the operating system and programming style. Currently there exist many different operating systems in which TinyOS and Contiki are most used and developed for. Both operating systems take quite different approaches to both operating system design and programming API. 

TinyOS is designed as a completely modular, event-driven and thread-less operating system which makes for a difficult system to program and reason about; this is due to its unfamiliar execution style and the design of its programming language, nesC, a significant variant of C. Due to its modular design, programs are written in ``components'' which are comprised of their ``configuration''(how they link/wire to other components) and ``module''(the implementation of the component). 

In contrast, Contiki is designed to fit the middle ground between Event driven programming, like TinyOS, and imperative programming, like C. It achieves this by having both lightweight threads, which can interleave in execution, and events, which allow the program to react to stimuli, written in a slight variation of C \cite{ContikiPaper}. This approach is much more familiar to the experienced programmer of larger systems whilst still providing an efficient environment in which to program albeit with some compromises (stackless threads). Where TinyOS breaks programs down into a complex array of components, modules and configurations, Contiki aligns with standard C, using header and .c files.

In regards to the ``Internet of Things'', Contiki provides a much more approachable platform to design for, as a significant portion of code written for larger platforms can be ported without much transformations to the code. Combining this with the hardware's array of sensors and low-power radio, the two create an ideal low end platform for developing at the bottom end of the ``Internet of Things''.


\begin{comment}

Some research 
popular academic tool, many OS's, very low power like Arduino, with radio and sensors
some gpio, but limited
Contiki c like, event and thread driven
\end{comment}
% subsection telos_b_motes (end)
% section open_source_constrained_devices (end)
\subsection{Wireless Sensor Networks} % (fold)
\label{sec:wireless_sensor_networks}
With computers becoming smaller, cheaper and more powerful every year, wireless sensor networks have become an area of rapid research. Wireless sensor networks allow the autonomous collection, aggregation and processing of data from hundreds if not thousands of devices, which before would have only been feasible through simulation. Use cases of such networks range from detecting forest fires\cite{FireDetection}, to monitoring the health of the Golden Gate bridge\cite{GoldenGate}.

Many of the situations and environments these networks are deployed in can be extremely harsh, often subjected to adverse weather conditions and even animal wildlife. Because of this, reliability and fault tolerance are significant areas of research, ensuring the network of devices can react and adapt to changes in the environment to ensure correct operation, even in the case of device failures. Due to these often harsh and changing environments, standard architectures of networking aren't feasible, e.g., star topology, because of their reliance on central points or base stations to communicate, which may become out of range or fail entirely; instead a more ad-hoc approach is needed to ensure all devices can be connected to the network, even when devices move in the physical environment. To solve this problem, devices are networked together in a mesh topology, where each device connects to its neighbours and allows traffic to route through one another to reach the final destination.

In some ways the ``Internet of Things'' can be interpreted as an extension or specialisation of wireless sensor networks, taking the autonomous network of hundreds of sensing devices and creating an autonomous closed-loop system, which can make intelligent decisions based on the environment, and placing it within a home, office or even city. Often the environments are a much less extreme and are more predictable than traditional wireless sensor network deployments, such as the home or office, reducing the concern for reliability. However, because of the pre-existing and often immutable nature of environments such as the home, installing traditional networking topologies ranging from the front door, to the second floor bedroom and all the way down to the bottom of the garden can pose problems; but like in wireless sensor networks, mesh topologies can help solve this by routing traffic through other nearby nodes, rather than trying to connect directly to the base station or router.

% section wireless_sensor_networks (end)

\section{Existing systems/protocols} % (fold)
\label{sec:existing_systems_protocols}

\subsection{Java JMS} % (fold)
\label{sub:java_jms}
Although the Java Messaging Service itself isn't directly targeted towards ``Internet of Things'' implementation, it does provide a standard and well defined framework for communicating and coordinating between both local and remote applications on a network. 

It can operate in two modes, either in a publish - subscribe or point-to-point architecture. Using the publish subscribe system, publishers publish to topics, to which subscribers subscribe. The topics bridge the two types of clients together and allow for many to many relationships without either side knowing about the other. With the ``Internet of Things'' in mind, this type of publish - subscribe systems fits in well with the \textit{if \textbf{event occurs} then \textbf{do action}} model. Topics map to events/conditions to which sensors can publish and to which other devices can subscribe, performing some action as a result, as shown in figure \ref{fig:JMS} .

\begin{figure}[h!]
	\centering
		\includegraphics[scale=0.4]{images/JMS-IoT.jpg}
		\captionof{figure}{JMS Publish - Subscribe modelling IoT}\label{fig:JMS}
\end{figure}


Whilst JMS provides a well fitted framework, it does however come at a cost due to the runtime environment of Java and the additional overhead of having to run a separate server to glue the publishers and subscribers together. Whilst currently running the Java runtime on these constrained devices may not be possible, with the current pace of innovation, especially in optimisation of the Java runtime, and Moore's Law giving devices more power for the same cost and form factor every 18 months, it won't be long before it's really possible to have it running on the constrained devices of tomorrow. But until then, other alternatives and perhaps more suited tools, libraries and frameworks should be engineered starting at the lowest common denominator rather than shoe-horning large scale systems to fit constrained devices.


\begin{comment}
A fully featured pub sub system as well as point to point
pubs and subs don't know about each other but interact through Topics
Topics emulate the concept of a service very well i.e. Topic = bedroom light, living room thermostat. where subscribers can subscribe to events published to their topic

Great for large, high power machines.
Not so great on small, low power devices
JVM can be scaled down to lower power devices but still poses overhead to system.
\end{comment}
% subsection java_jms (end)

\subsection{xAP} % (fold)
\label{sub:xap}

In the early 2000's, as an attempt to bring together a variety of technologies developed and used for Home Automation, an open source group decided to create a protocol to bridge the differences and create a unifying platform.\cite{xAP,xAProtocol}

xAP, eXtensible Automation Protocol is designed to be a minimalist, elegant and easy to implement protocol with very basic requirements for the hardware, operating system, network and language on which it can run.  
The primary implementation is based on UDP/IP with a distributed architecture where no central controller co-ordinates the network. Instead, each device either transmits or listens for data on a broadcast channel. Their core justification for this is that the network becomes fault tolerant and can withstand devices disappearing off the network for one reason or another without having a detrimental effect on the whole network.

In this design there are two key classes of devices, senders and receivers, of which a device can be either or both. 
The receiver simply attaches to the network and listens for any incoming packets, it then chooses which ones are relevant to it and processes them.
The sender attaches to the network and broadcasts packets with payloads containing data relating to its service i.e., sensor data, incoming caller ID, etc. These packets only contain enough information to uniquely identify the source and the payload it wishes to send. There is no constraint on whether or not the packet must have a destination.

This design feature makes it very simple for new devices to attach to the network and start interacting without having to go through the process of setting up connections to other devices. It also makes it very easy to implement certain types of devices like loggers or informational displays which can collect data from all senders with relative ease. However, this also creates a huge strain on the devices attached to the network which all have to receive and process any broadcasted data, especially for those devices with limited power resources. This becomes an increasing problem as more and more connected devices invade the home, from which the network will very quickly deteriorate as the volume of traffic being broadcast increases. The efficient sending of data, by either avoiding broadcast or limiting its use is an ongoing research topic within the Wireless Sensor Networks domain, due to the costly nature of broadcast and the constrained power resources of the WSN devices.\cite{RumourRouting}\cite{DirectedDiffusion}

The protocol also defines a set schema for how packets should be formatted so that all devices can interoperate correctly. The schema covers both the header and the payload format, which is almost entirely text-based, resembling, albeit pre-dating, JSON, with the main intention of being human readable. Whilst so, this does come at the cost of not only increasing the size of the packet considerably with redundant text, but also the complexity in parsing the data with its varying delimiters. 

After the protocol's initial inception it has had limited success, even whilst it has been continuously developed over the years, it has done so in the shadows and not in an explicitly open and collaborative way, with no central repository for code collaboration/review and no single developer's forum for discussion. % POST LINK TO YAHOO FORUMS ETC

% subsection xap (end)

\subsection{Remote Procedure Call - RPC} % (fold)
\label{sub:rpc}
RPC whilst relatively old in comparison to the other systems described, it does provide an adequate abstraction for communication between devices in an ``Internet of Things'', via request/response style interactions. 

The concept of creating a distributed system, whereby devices in the network request data and services from each on demand, with the relevant response being returned. This fits in well with the idea of an autonomous system sensing, processing and reacting to stimuli, with each device bringing its own services to the system.

However, because of the fixed concept of every request requires a response, the number of packets sent is doubled, even when no response is necessary. In the environment of constrained devices, this increases the load per device and if battery powered, reduces the life of the device significantly.
% subsection rpc (end)
% section existing_systems_protocols (end)

\section{Developing systems/protocols} % (fold)
\label{sec:developing_systems}

\subsection{OpenWSN} % (fold)
\label{sub:owsn_berkeley}
Currently being developed at the University of California, Berkeley, the Open Wireless Sensor Network project is collection of open-source implementations of protocol stacks implemented against to-be-finalised ``Internet of Things'' standards, for a range of different software and hardware platforms.

The driving force behind such a project is to create an open source implementation protocols, which academia and industry can use to conduct further research, develop standardised products and contribute to an eco-system of ``Internet of Things'' devices, rather than researching and developing proprietary devices. In the long run this can also benefit consumers, increasing their choice of devices across many different and compatible brands/manufacturers.

Whilst the OpenWSN project doesn't actually propose any new research or developments by itself, it does pursue the same idea as this very project, trying to create an open source method in which heterogeneous devices can communicate with one another to form and ``Internet of Things''.

The OpenWSN ``Internet of Things'' stack consists of open source implementations of the proposed communications stack, ranging from the physical layer (IEEE802.15.4-2006), to the adaptation of IPv6 (IETF 6LoWPAN) all the way up to the application layer, of which is of most interest. The application layer implements not only HTTP, but also a currently in progress standard called Constrained Application Protocol, with the purpose of bringing RESTful communication to constrained devices.



\subsection{CoAP} % (fold)
\label{sub:coap}
CoAP, Constrained Application Protocol is currently a proposed IETF standard to create a request-response style web protocol, specifically designed to fit the limitations of constrained devices, such as micro-controllers, for use in machine-to-machine communication and the``Internet of Things''. The protocol is designed as an asynchronous RESTful protocol, to run on top of UDP, using a subset of the standard commands e.g., GET, PUT, DELETE. The reason is that due to the common commands, it allows for an easy transition to standard HTTP if necessary, whilst reducing the complexity of the protocol. Together with this, the protocol also provides support for service discovery and basic subscribing to other CoAP nodes.

Because the protocol is designed to run on top of UDP, which is a connectionless, unreliable transport layer, CoAP provides both un-reliable and reliable support for requests; which is provided by marking a message as non-confirmable (unreliable) or confirmable (reliable). This allows nodes to send data which is essential, reliably, and to send data un-reliable when data is ephemeral, such as sensor readings.   

As mentioned previously, the protocol also supports service discovery, to allow nodes to find new sources to request resources from, without necessarily using a centralised server. If a node's IP address is known a priori, then a unicast discovery can be used to locate the entry point of the resource of interest, otherwise a multicast discovery must be used; this is done by making a GET request to a standardised location, from which receiving nodes respond based on whether the request matches any of the resources it holds (by type or description).\cite{IETF_CORE}

To combat one of the problems normally associated with constrained devices, power, CoAP supports the ability for nodes to subscribe to one another. This means that instead of a node being repeatedly polled for any changes, the node instead notifies all subscribed nodes when the resource of interest has changed. This reduces the amount of packets received by the receiving node, in turn helping to reduce its power consumption.

Lastly, because CoAP is so similar to HTTP in the commands that it uses (GET, PUT, DELETE, etc.), it makes it simple to translate messages between protocols, and therefore allow these constrained devices to send and receive data to and from the WWW.\cite{IETF_COAP_HTTP} 

At the time of writing, the protocol is still being developed and finalised, although there have been a variety of implementations (of some degree) \cite{COAP_1}\cite{COAP_2}\cite{COAP_3}, it's yet to see if the protocol will be fully adopted both commercially and in the open source community.

% subsubsection coap (end)
% subsection owsn_berkeley (end)

\subsection{SmartThings} % (fold)
\label{sub:smart_things}
Launched in 2012, the SmartThings platform aims to allow the user to turn any ordinary object in the home into a ``Smart'' object by giving it the ability to connect to the Internet. The platform consists of a central hub connected to the Internet, containing a low-power Zigbee radio, from which it connects to an array of SmartThings accessories as well as many pre-existing third party Zigbee devices. Some of these accessories include motion detectors, moisture sensors, vibration detectors, power-plug switches as well as many others. The user can then connect to the SmartThings service through an Internet-connected pc or a smartphone, allowing them to either control the devices directly or set-up rules/schedules for devices, similar to the previously mentioned Twine device(\ref{sub:past_present_and_future}). 

Whilst providing easy to set-up pre-made SmartThings, the developer also provides the open-source Arduino kits to let hobbyists create their own SmartThings utilising the already well supported Arduino platform.

Currently, there is no information on how the underlying protocol for connecting the SmartThings to the hub works, but at the time of writing, the current implementation uses a ``cloud first'' approach. This means that rather than the hub wiring all devices together based on the rules and schedules set up by the user, all the intelligence of the network is being handled in the cloud. This brings about the problem of Internet connectivity, with two points of failure, either the user or the cloud. From the user's standpoint, an Internet connection might not be available where the hub is located, or the user could have a faulty, slow or non-permanent connection, which renders all the SmartThings devices into dumb devices. In contrast to this, because of the reliance on the cloud, if the cloud service provider experiences downtime then all SmartThings users devices become dumb devices. In cases where these devices are used for security and safety, dire consequences could result.
  


% subsection smart_things (end)

\subsection{Qualcomm} % (fold)
\label{sub:qualcomm}
As an example of a major hardware manufacturer acknowledging the ``Internet of Things'', at CES 2013, Qualcomm announced a new platform for developing ``Things'', named the ``Internet of Everything'', based on their mobile chipsets, providing 3G network connectivity and utilising the Java Micro Edition runtime \cite{Qualcomm}. So far it seems the concept of connectivity is not dissimilar to existing devices, such as the Twine, where the device is essentially a single device connecting to the Internet, in contrast to a network of interconnected ``Things''.

The platform has yet to be released at the time of writing, but will prove to be an interesting start to hopefully more manufacturers creating purposely designed hardware, enabling developers to create ``Internet of Things'' applications.

% subsection qualcomm (end)
% section developing_systems (end)


