The ``Internet of Things'' paradigm, along with the ``Smart'' prefix, has recently seen a significant rise in interest and popularity with manufacturers, hobbyists and end-users. Everything from your set-top box to your washing machine can now be ``Smart'' and connect to the Internet to tell you if your favourite TV show has downloaded or that your wash cycle has complete\cite{LG, SmartCraze}. Some hobbyists have gone so far as to make a plant send a text or tweet when it needs watering\cite{Botanicalls, TweetPlant}.

This uptake in interest and development can be largely attributed to the advent of lower power, smaller and cheaper devices. Large companies, such as the mobile phone chip-set manufacturer Qualcomm, recently declared their support at CES 2013 with the announcement of a dedicated development platform for the ``Internet of Things''\cite{Qualcomm}.

Similarly there has also been significant enthusiasm from small companies and start-ups trying to create the next hit consumer device. The social funding platform, Kickstarter\cite{Kickstarter}, has seen many attempts at creating the perfect ``Thing'' for the Internet\cite{SmartThings, Twine}.

Whilst all of these devices from manufacturers, start-ups and hobbyists may very well be great ``Things'' by themselves, there exists a problem. How does one connect all of these heterogeneous platforms and devices together to create a truly interconnected ``Internet of Things''?

This project focuses on just that and endeavours to create a communication protocol that is not only platform agnostic but also lightweight enough to be run on the most constrained devices, such as the TelosB Sky Motes(8MHz CPU,10K ram)\cite{TelosB}. Another core design focus is that the protocol must be able to scale effectively as the network dramatically increases size as can be expected with the rapidly increasing availability of network-connected devices.