\pdfoutput=1

\documentclass{l4proj}

%
% put any packages here
%


\begin{document}
\title{Level 4 Project: \\
		Designing a lightweight protocol for constrained devices for use in the Internet of Things paradigm}
\author{Fergus Leahy}
\date{}
\maketitle

\begin{abstract}
A level 4 project which explores the current use of protocols for the internet of things devices.
\end{abstract}

\educationalconsent
%
%NOTE: if you include the educationalconsent (above) and your project is graded an A then
%      it may be entered in the CS Hall of Fame
%
\tableofcontents
%==============================================================================

\chapter{Introduction}
\pagenumbering{arabic}

\subsection{Prospect} % (fold)
\label{sub:prospect}

% subsection prospect (end)

\subsection{Aim} % (fold)
\label{sub:aim}

% subsection aim (end)
%==============================================================================


%==============================================================================

\chapter{Background} % (fold)
\label{cha:background}

\section{Internet of Things paradigm} % (fold)
\label{sec:internet_of_things_paradigm}

% section internet_of_things_paradigm (end)

\section{Open source constrained devices} % (fold)
\label{sec:open_source_constrained_devices}
\subsection{Arduino} % (fold)
\label{sub:arduino}

% subsection arduino (end)

\subsection{Raspberry Pi} % (fold)
\label{sub:raspberry_pi}

% subsection raspberry_pi (end)
% section open_source_constrained_devices (end)
\section{Wireless Sensor Networks} % (fold)
\label{sec:wireless_sensor_networks}

% section wireless_sensor_networks (end)

\section{Existing systems/protocols} % (fold)
\label{sec:existing_systems_protocols}

\subsection{Java JMS} % (fold)
\label{sub:java_jms}

% subsection java_jms (end)

\subsection{xAP} % (fold)
\label{sub:xap}
\cite{xAP}
% subsection xap (end)

\subsection{Developing systems} % (fold)
\label{sub:developing_systems}

\subsubsection{Smart things} % (fold)
\label{ssub:smart_things}

% subsubsection smart_things (end)

\subsubsection{Qualcomm} % (fold)
\label{ssub:qualcomm}

% subsubsection qualcomm (end)
% subsection developing_systems (end)
% section existing_systems_protocols (end)



% chapter background (end)
%==============================================================================


%==============================================================================

\chapter{Requirments gathering} % (fold)
\label{cha:requirments_gathering}

% chapter requirments_gathering (end)
%==============================================================================


%==============================================================================


\chapter{Design} % (fold)
\label{cha:design}

% chapter design (end)
%==============================================================================


%==============================================================================


\chapter{Implementation} % (fold)
\label{cha:implementation}

% chapter implementation (end)
%==============================================================================


%==============================================================================


\chapter{Evaluation} % (fold)
\label{cha:evaluation}

% chapter evaluation (end)

%==============================================================================


%==============================================================================


\chapter{Conclusion} % (fold)
\label{cha:conclusion}


% chapter conclusion (end)
%==============================================================================


%%%%%%%%%%%%%%%%
%              %
%  APPENDICES  %
%              %
%%%%%%%%%%%%%%%%
\begin{appendices}

\chapter{Running the Programs}
An example of running from the command line is as follows:
\begin{verbatim}
      > java MaxClique BBMC1 brock200_1.clq 14400
\end{verbatim}
This will apply $BBMC$ with $style = 1$ to the first brock200 DIMACS instance allowing 14400 seconds of cpu time.

\chapter{Generating Random Graphs}
\label{sec:randomGraph}
We generate Erd\'{o}s-R\"{e}nyi random graphs $G(n,p)$ where $n$ is the number of vertices and
each edge is included in the graph with probability $p$ independent from every other edge. It produces
a random graph in DIMACS format with vertices numbered 1 to $n$ inclusive. It can be run from the command line as follows to produce 
a clq file
\begin{verbatim}
      > java RandomGraph 100 0.9 > 100-90-00.clq
\end{verbatim}
\end{appendices}

%%%%%%%%%%%%%%%%%%%%
%   BIBLIOGRAPHY   %
%%%%%%%%%%%%%%%%%%%%

\bibliographystyle{plain}
\bibliography{bib}

\end{document}
