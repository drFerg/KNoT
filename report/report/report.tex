\pdfoutput=1

\documentclass{l4proj}
\usepackage{comment}
%
% put any packages here
%


\begin{document}
\title{Designing a lightweight protocol for constrained devices for use in the Internet of Things paradigm}
\author{Fergus Leahy}
\date{\today}
\maketitle

\begin{abstract}
\textbf{PLACEHOLDER}
A level 4 project which explores the current use of protocols for the Internet of things devices.
\end{abstract}

\educationalconsent
%
%NOTE: if you include the educationalconsent (above) and your project is graded an A then
%      it may be entered in the CS Hall of Fame
%
\tableofcontents
%==============================================================================

\chapter{Introduction}
\pagenumbering{arabic}

The ``Internet of Things'' paradigm along with the ``Smart'' prefix has recently seen a significant rise in interest and popularity with manufacturers, hobbyists and end-users. Everything from your set-top box to your washing machine\cite{LG} can now be ``Smart''\cite{SmartCraze} and connect to the Internet to tell you if your favourite TV show has downloaded or that your wash cycle has complete. Some hobbyists have gone so far to make plants send a text or tweet when it needs watering\cite{Botanicalls}\cite{TweetPlant}.

This uptake in interest and development can be largely attributed to the advent of lower power, smaller and most importantly cheaper devices. Large companies such as the mobile phone chip-set manufacturer, Qualcomm, recently declared their support at CES 2012 with the announcement of a dedicated development platform for the ``Internet of Things''\cite{Qualcomm}.

Similarly there has also been much enthusiasm from small companies and start ups trying to create the next hit consumer device. Social funding platform, Kickstarter\cite{Kickstarter}, has seen many attempts at creating the perfect ``Thing'' for the Internet\cite{SmartThings}\cite{Twine}.

Whilst all of these devices from manufacturers, start ups and hobbyists may very well be great ``Things'' by themselves, there exists a problem. How do you connect all of these heterogeneous platforms and devices together to create a true Internet of ``Things''?

This project focuses on just that and endeavours to create a communication protocol that is not only platform agnostic but also lightweight enough to be run on the most brain-dead constrained devices, such as the TelosB Sky Motes(8MHz,10K ram)\cite{TelosB}. Another design focus is that the protocol must be able to scale well as the network dramatically increases and decreases in size.

%==============================================================================

\chapter{Background and Related Work} % (fold)
\label{cha:background}

\section{Internet of Things paradigm} % (fold)
\label{sec:internet_of_things_paradigm}


The standard model of how we use computers and the Internet today revolves around the idea of an interconnected network of servers, routers and data centres around which users connect using their personal computers to access, input, manipulate and retrieve data. 

This model heavily relies upon the users to provide the data which the Internet is powered by. Without this data the Internet would be a barren place, with nothing to search for, sell, buy, share, watch, listen to, play, analyse or learn from.
Users from all around the world have contributed to make the Internet \textit{the} single biggest resource in the world by snapping photos, capturing videos, writing blogs, creating websites, commenting, discussing, reviewing, buying, selling and inputing data. So much data in fact that the estimated size of the Internet in 2009 was 500bn gigabytes, of which 70\% was contributed to by users.\cite{Size}   

% INSERT DIAGRAM HERE
Within this model there exists two significant problems posed by users; time and accuracy. In terms of time, users only have so many hours in a day to input data; which can only be so accurate as all users are prone to error through one means or another.
These two problems make it difficult to observe our world and represent it in an accurate and reliable digital form.

The idea of taking this responsibility of inputing data away from the user and giving it to the machine is where the ``Internet of Things'' term initially came from. The term itself was thought to have first been coined by Kevin Ashton\cite{K.Ashton} in 1999, who at the time was interested in linking up a company's supply chain to the Internet using RFID technology to allow for autonomous monitoring its state.
Sadly at that time the idea progressed little and didn't garner much support.

Fast forward to 2013 and the ``Year of the Internet of Things'' as declared by the MIT Technology Review\cite{2013IoT} and observed in the Consumer Electronics Show 2013 (CES).

 % Kevin Ashton, RFID, coined IoT
 % http://www.rfidjournal.com/article/view/4986
 % Giving everyday objects a digital presence and creating machine to machine connections
 % Autonomy, semi-autonomy
 % Taking the human out of the equation and empowering the computer to produce and consume its own data to make our world more 
 % efficient, less wasteful, more time and less energy.
 % Slow start, not much going on.

 % 2012/13 jump start, lots of manufacturers
 % open hardware initiatives
 % linking home automation, security and IoT i.e. toaster/coffee machine
 % 


% section internet_of_things_paradigm (end)
\newpage
\section{Open source constrained devices} % (fold)
\label{sec:open_source_constrained_devices}

\subsection{Arduino} % (fold)
\label{sub:arduino}
In the past decade there have been many attempts at creating an affordable, low power and approachable electronics platform such as Arduino, Netduino, BeagleBone, Teensy and MSP430 Launchpad. All of which have taken very different approaches, some opting for absolute low cost (MSP430 Launchpad), whilst others for aimed to be fully featured and powerful devices (BeagleBone). Whilst the Arduino is by far not the cheapest or the most powerful of the bunch, it excelled in areas others lacked and created a sweet spot in the middle.

The Arduino was born in 2005 at an Italian university, Interaction Design Institute Ivrea, out of the necessity of creating a cheap and approachable electronics platform which could enable design students to create interaction design projects without the need of an electronics background.
The main device which was created and remained much the same since is based on an Amtel ATmega328 8bit micro-controller running at 16MHz with 2KB of RAM and 32KB of storage for programmes written in a variation on C/C++. The board itself maps out the micro-controller's mix of 20 digital and analog input/output pins and supports several standardised protocols for communicating with other devices such as I\textsuperscript{2}C\footnote{I\textsuperscript{2}C - Inter-Intergrated Circuit} and UART\footnote{UART - Universal Asynchronous Reciever/Transmitter}. 

But the key to the Arduino Platform is not the micro-controller itself but instead the design, software and support provided by the creators and other developers. The other major factor to its success is that the device along with the documentation and support are all open source, thus allowing anyone to learn from, replicate and expand upon the platform in any way they wish.

An example of how these factors have had a hugely positive effect is something which Arduino calls ``Shields''. These shields plug in on top of the Arduino board and contain standard components which can add additional features such as WI-FI, Ethernet, sound, motor control etc. Rather than users being required to find, purchase and solder the required components to add such features, these pre-built shields provide it in simple and readily available package made by a variety of manufacturers.

Since its creation the Arduino platform has created a range of Arduino named devices and shields resulting in a following of over 300,000 users\cite{ArduinoNumbers} and support from many manufacturers and distributors worldwide. 

Due to Arduino's open policy many new start-ups have been able to quickly design prototypes and products using the Arduino platform which have been taken to market in various refined forms. Often the same micro-controller which powers the Arduino is kept and the board miniaturised to fit the product's needs. Products such as the Internet ``Thing'' Twine\cite{Twine} and the Smart Things ``Internet of Things'' eco-system have taken this approach\cite{SmartThings}.

However, whilst there is an abundance of Arduino devices in the wild with many being used as an Internet ``Thing'' there is yet to be created an open and compatible method to easily network a group of such devices together to form a connected ``Internet of Things'' network.

\begin{comment}
OPEN SOURCE
Micro-controller, cheap, easy, PIC chips were difficult, lowered barrier of entry, provided large support and huge following
Italian made, several iterations on size and power
Good starting point for development as there are many in existence
Well adopted
Based on C/C++ with a few tweaks, offers shields for expandability i.e. Ethernet, WI-FI
very good at sensing and doing things
16MHz
no threading
\end{comment}
% subsection Arduino (end)

\newpage
\subsection{Raspberry Pi} % (fold)
\label{sub:raspberry_pi}
Launched in 2012 the Raspberry Pi, a \$35 credit card sized computer, was eagerly anticipated to change the computing landscape. The charity behind it, the Raspberry Pi foundation, had one main goal; to refresh and promote the teaching of computer science in schools.

Similar to the Arduino Platform most of the hardware and software for the device is open source with the aim of letting adults and children alike get their hands dirty learning about computing without the worry of breaking an expensive computer.

The Raspberry Pi itself is a moderately well powered single board computer capable of running a wide variety of Linux based operating systems. It's host to a 700MHz ARM CPU, 512MB RAM and a variety of inputs/outputs including an Ethernet port. These features along with some additional GPIO\footnote{General Purpose Input Output} pins, just like the Arduino, allow it to sense and control the world around it, thus making it a perfect ``Thing''.

Because of extremely low price point for such hardware the Raspberry Pi was a huge success selling over 500,000 units in the first 8 months only limited by their production speed.\cite{RaspberryPiSold}

Even before it's release the community support and ideas were non-stop, everything from home-media centres\cite{Raspbmc} to lego super computers\footnote{Even within the School of Computing, UoG}\cite{LegoSuperComputer} were designed and created from Raspberry Pis.

Whilst the support for utilising the GPIO pins is not yet as well developed as the Arduino Platform, the power of the device combined with its low price point and connectivity means that it can be very easily incorporated into a ``Internet of Things'' network with little additional hardware and cost.

% ADD PROGRAMMING LANGUAGE PART

\begin{comment}
new, more powerful open device. runs a real operating system with gpio options
perhaps upper limit of power for this protocol
most languages possible, threading etc
\end{comment}
% subsection raspberry_pi (end)

\subsection{Telos B Motes} % (fold)
\label{sub:telos_b_motes}

With a significant uptake of devices in the academic ..., telos B motes are one of the go to platforms for wireless sensor networks.
Whilst they may fit under a different paradigm name, the IoT can be seen as an expansion and specialisation of a wireless sensor network.
With this in mind these Telos B motes are a very good fit for such an environment. 

The devices themselves were developed in the University of Berkely, California and spun off into a seperate company. These devices have fuelled many academic studies/papers and courses in relation to developing low power wireless sensor networks and finding applications for them.
GIVE EXAMPLES OF APPLICATIONS
DEVICE SPECS

A variety of very different OS's with stark differences in programming styles.

\begin{comment}

Some research 
popular academic tool, many OS's, very low power like Arduino, with radio and sensors
some gpio, but limited
Contiki c like, event and thread driven
\end{comment}
% subsection telos_b_motes (end)
% section open_source_constrained_devices (end)
\section{Wireless Sensor Networks} % (fold)
\label{sec:wireless_sensor_networks}
Discuss relation to wireless sensor networks, similar disimilar
Concerns which differ, align.

% section wireless_sensor_networks (end)

\section{Existing systems/protocols} % (fold)
\label{sec:existing_systems_protocols}

\subsection{Java JMS} % (fold)
\label{sub:java_jms}
Great for large, high power machines.
Not so great on small, low power devices
JVM can be scalled down but still poses overhead to system.
% subsection java_jms (end)

\subsection{xAP} % (fold)
\label{sub:xap}
\cite{xAP}
Nice concepts and goals
BUT fails to deliver a system which can scale and do things efficiently

% subsection xap (end)

\subsection{Developing systems} % (fold)
\label{sub:developing_systems}

\subsubsection{Smart things} % (fold)
\label{ssub:smart_things}

% subsubsection smart_things (end)

\subsubsection{Qualcomm} % (fold)
\label{ssub:qualcomm}
2013 CES Discussed bringing mobile platform hardware to IOT, but still creating a single device with internet as hub mainly 
% subsubsection qualcomm (end)
% subsection developing_systems (end)
% section existing_systems_protocols (end)



% chapter background (end)
%==============================================================================


%==============================================================================

\chapter{Requirments gathering} % (fold)
\label{cha:requirments_gathering}

% Requirements of the hardware independent design

% section problems_encountered (end)
% chapter requirments_gathering (end)
%==============================================================================


%==============================================================================


\chapter{Design} % (fold)
\label{cha:design}

% Design of system independent of hardware implementation.

\section{Protocol Design} % (fold)
\label{sec:protocol_design}
Minimal protocol, only ack required/critical data to save on transmissions
Provide enough garuantees for a useful and reliable system which can scale
Provide a distributed and fault tolerant way to find, collect data and control devices from



\subsection{Protocol Data Unit} % (fold)
\label{sub:protocol_data_unit}

% subsection protocol_data_unit (end)

\subsection{Payload Formats} % (fold)
\label{sub:payload_formats}

\subsubsection{Query} % (fold)
\label{ssub:query}
		
% subsubsection query (end)

\subsubsection{Query Response} % (fold)
\label{ssub:query_response}

% subsubsection query_response (end)

\subsubsection{Connect} % (fold)
\label{ssub:connect}

% subsubsection connect (end)

\subsubsection{Connect ACK} % (fold)
\label{ssub:connect_ack}

% subsubsection connect_ack (end)

\subsubsection{Response} % (fold)
\label{ssub:response}

% subsubsection response (end)

\subsubsection{Ping} % (fold)
\label{ssub:ping}

% subsubsection ping (end)

\subsubsection{Ping ACK} % (fold)
\label{ssub:ping_ack}

% subsubsection ping_ack (end)

\subsubsection{Disconnect} % (fold)
\label{ssub:disconnect}

% subsubsection disconnect (end)

\subsubsection{Disconnect ACK} % (fold)
\label{ssub:disconnect_ack}

% subsubsection disconnect_ack (end)
% subsection payload_formats (end)


% section protocol_design (end)




% chapter design (end)
%==============================================================================


%==============================================================================


\chapter{Implementation} % (fold)
\label{cha:implementation}
% Describe the problems encountered in the implementation of arduino and then contiki

\section{Arduino} % (fold)
\label{sec:arduino}
Whilst given its user base and ideology makes it a great platform to build for, the immediate lack of compatible hardware and lack of certain expectations leaves it difficult to develop an efficient implementation of the system.

No interrupts
% section arduino (end)


\section{Telos B Motes and Contiki} % (fold)
\label{sec:contiki}
Great, well defined and established hardware.
Contiki is a comprehensive and well devised OS which provides a simple learning curve for a C programmer with only some event magic thrown in.

\subsection{Choosing the transport layer:IP4 vs IPv6 vs Rime} % (fold)
\label{sub:ip4_vs_ipv6_vs_rime}

% subsection ip4_vs_ipv6_vs_rime (end)

\subsection{Problems Encountered} % (fold)
\label{sub:problems_encountered}

\subsubsection{Event driven system} % (fold)
\label{sub:event_driven_system}

% subsubsection event_driven_system (end)

\subsubsection{IPv4 Broadcast support} % (fold)
\label{sub:ipv4_broadcast_support}

% subsubsection ipv4_broadcast_support (end)
% subsection problems_encountered (end)

% section contiki (end)
% chapter implementation (end)
%==============================================================================


%==============================================================================


\chapter{Evaluation} % (fold)
\label{cha:evaluation}

% chapter evaluation (end)

%==============================================================================


%==============================================================================


\chapter{Conclusion} % (fold)
\label{cha:conclusion}


% chapter conclusion (end)
%==============================================================================


%%%%%%%%%%%%%%%%
%              %
%  APPENDICES  %
%              %
%%%%%%%%%%%%%%%%
\begin{appendices}


\end{appendices}

%%%%%%%%%%%%%%%%%%%%
%   BIBLIOGRAPHY   %
%%%%%%%%%%%%%%%%%%%%

\bibliographystyle{plain}
\bibliography{bib}

\end{document}
