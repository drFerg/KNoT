\pdfoutput=1

\documentclass{l4proj}

%
% put any packages here
%


\begin{document}
\title{Designing a lightweight protocol for constrained devices for use in the Internet of Things paradigm}
\author{Fergus Leahy}
\date{\today}
\maketitle

\begin{abstract}
\textbf{PLACEHOLDER}
A level 4 project which explores the current use of protocols for the Internet of things devices.
\end{abstract}

\educationalconsent
%
%NOTE: if you include the educationalconsent (above) and your project is graded an A then
%      it may be entered in the CS Hall of Fame
%
\tableofcontents
%==============================================================================

\chapter{Introduction}
\pagenumbering{arabic}

The ``Internet of Things'' paradigm along with the ``Smart'' prefix has recently seen a significant rise in interest and popularity with manufacturers, hobbyists and end-users. Everything from your set-top box to your washing machine\cite{LG} can now be ``Smart''\cite{SmartCraze} and connect to the Internet to tell you if your favourite TV show has downloaded or that your wash cycle has complete. Some hobbyists have gone so far to make plants send a text or tweet when it needs watering\cite{TweetPlant}.

This uptake in interest and development can be largely attributed to the advent of lower power, smaller and most importantly cheaper devices. Large companies such as the mobile phone chip-set manufacturer, Qualcomm, recently declared their support at CES 2012 with the announcement of a dedicated development platform for the ``Internet of Things''\cite{Qualcomm}.

Similarly there has also been much enthusiasm from small companies and start ups trying to create the next hit consumer device. Social funding platform, Kickstarter\cite{Kickstarter}, has seen many attempts at creating the perfect ``Thing'' for the Internet\cite{SmartThings}\cite{Twine}.

Whilst all of these devices from manufacturers, start ups and hobbyists may very well be great ``Things'' by themselves, there exists a problem. How do you connect all of these heterogeneous platforms and devices together to create a true Internet of ``Things''?

This project focuses on just that and endeavours to create a communication protocol that is not only platform agnostic but also lightweight enough to be run on the most brain-dead constrained devices, such as the TelosB Sky Motes(8MHz,10K ram)\cite{TelosB}. Another design focus is that the protocol must be able to scale well as the network dramatically increases and decreases in size.

%==============================================================================

\chapter{Background and Related Work} % (fold)
\label{cha:background}

\section{Internet of Things paradigm} % (fold)
\label{sec:internet_of_things_paradigm}


The standard model of how we use computers and the Internet today revolves around the idea of an interconnected network of servers, routers and data centres around which users connect using their personal computers to access, input, manipulate and retrieve data. 

This model heavily relies upon the users to provide the data which the Internet is powered by. Without this data the Internet would be a barren place, with nothing to search for, sell, buy, share, watch, listen to, play, analyse or learn from.
Users from all around the world have contributed to make the Internet \textit{the} single biggest resource in the world by snapping photos, capturing videos, writing blogs, creating websites, commenting, discussing, reviewing, buying, selling and inputing data. So much data in fact that the estimated size of the Internet in 2009 was 500bn gigabytes, of which 70\% was contributed to by users.\cite{Size}   

% INSERT DIAGRAM HERE
Within this model there exists two significant problems posed by users; time and accuracy. In terms of time, users only have so many hours in a day to input data; which can only be so accurate as all users are prone to error through one means or another.
These two problems make it difficult to observe our world and represent it in an accurate and reliable digital form.

The idea of taking this responsibility of inputing data away from the user and giving it to the machine is where the ``Internet of Things'' term initially came from. The term itself was thought to have first been coined by Kevin Ashton\cite{K.Ashton} in 1999, who at the time was interested in linking up a company's supply chain to the Internet using RFID technology to allow for autonomous monitoring its state.
Sadly at that time the idea progressed little and didn't garner much support.

Fast forward to 2013 and the ``Year of the Internet of Things'' as declared by the MIT Technology Review\cite{2013IoT} and observed in the Consumer Electronics Show 2013 (CES).

 % Kevin Ashton, RFID, coined IoT
 % http://www.rfidjournal.com/article/view/4986
 % Giving everyday objects a digital presence and creating machine to machine connections
 % Autonomy, semi-autonomy
 % Taking the human out of the equation and empowering the computer to produce and consume its own data to make our world more 
 % efficient, less wasteful, more time and less energy.
 % Slow start, not much going on.

 % 2012/13 jump start, lots of manufacturers
 % open hardware initiatives
 % linking home automation, security and IoT i.e. toaster/coffee machine
 % 


% section internet_of_things_paradigm (end)

\section{Open source constrained devices} % (fold)
\label{sec:open_source_constrained_devices}
\subsection{Arduino} % (fold)
\label{sub:arduino}

% subsection arduino (end)

\subsection{Raspberry Pi} % (fold)
\label{sub:raspberry_pi}

% subsection raspberry_pi (end)
% section open_source_constrained_devices (end)
\section{Wireless Sensor Networks} % (fold)
\label{sec:wireless_sensor_networks}

% section wireless_sensor_networks (end)

\section{Existing systems/protocols} % (fold)
\label{sec:existing_systems_protocols}

\subsection{Java JMS} % (fold)
\label{sub:java_jms}

% subsection java_jms (end)

\subsection{xAP} % (fold)
\label{sub:xap}
\cite{xAP}
% subsection xap (end)

\subsection{Developing systems} % (fold)
\label{sub:developing_systems}

\subsubsection{Smart things} % (fold)
\label{ssub:smart_things}

% subsubsection smart_things (end)

\subsubsection{Qualcomm} % (fold)
\label{ssub:qualcomm}

% subsubsection qualcomm (end)
% subsection developing_systems (end)
% section existing_systems_protocols (end)



% chapter background (end)
%==============================================================================


%==============================================================================

\chapter{Requirments gathering} % (fold)
\label{cha:requirments_gathering}

% chapter requirments_gathering (end)
%==============================================================================


%==============================================================================


\chapter{Design} % (fold)
\label{cha:design}

% chapter design (end)
%==============================================================================


%==============================================================================


\chapter{Implementation} % (fold)
\label{cha:implementation}

% chapter implementation (end)
%==============================================================================


%==============================================================================


\chapter{Evaluation} % (fold)
\label{cha:evaluation}

% chapter evaluation (end)

%==============================================================================


%==============================================================================


\chapter{Conclusion} % (fold)
\label{cha:conclusion}


% chapter conclusion (end)
%==============================================================================


%%%%%%%%%%%%%%%%
%              %
%  APPENDICES  %
%              %
%%%%%%%%%%%%%%%%
\begin{appendices}


\end{appendices}

%%%%%%%%%%%%%%%%%%%%
%   BIBLIOGRAPHY   %
%%%%%%%%%%%%%%%%%%%%

\bibliographystyle{plain}
\bibliography{bib}

\end{document}
