% Requirements of the hardware independent design
 
% Looking at xAP, JMS, what's wrong, how can that be resolved
% what are typical use cases/ types of devices

As discussed in the previous chapter, many different platforms and protocols exist to create an ``Internet of Things'' and most take very different approaches. This chapter will highlight several drawbacks of some the previously mentioned systems, explain how such problems can be resolved with this new approach and discuss the requirements for developing a new system, which is better suited for constrained devices and the environment of the ``Internet of Things''.

\section{Primary Requirements}
\label{sec:primary_requirements}
Due to the nature of the target hardware platforms, constrained devices such as Arduino and TelosB, heavyweight approaches become infeasible i.e., Java \& JMS. Therefore an extremely lightweight and scalable protocol, both in terms of the protocol data unit (pdu) and the complexity in processing and managing the runtime, is a primary requirement.

Secondary to this, because of the often unstructured and ad-hoc design of a typical ``Internet of Things'' network, building a centralised system i.e., a name server for device search, is an unnatural fit with the network. A single failure in the server could bring the entire network down, whilst all other devices are fully operational and no way to communicate with each other.
Instead, creating a distributed system which gives all the devices in the network the power to discover and communicate with each other, relieves the network of a single point of failure and can help spread the load. % ADD SOMETHING TO DO WITH DISCOVERY

Taking into consideration the typical types of devices connected to the ``Internet of Things'', three general roles of devices can be discerned; a sensor, an actuator and a controller. Sensors, anything from a temperature sensor to an open door sensor, provide a continuous service to other devices in the form of real-time data, usually from the outside world. Similarly, actuators also provide a service to other devices in the form of an interaction with the outside world, such as a speaker, light switch or thermostat. Lastly, controllers are devices which orchestrate the ``Things'' in the network, forming relationships between devices and creating useful interactions between the digital and real world i.e., connecting to both a door sensor and a light switch, when the door opens the light turns on. Whilst there are these distinct roles, it is often the case that one device could take on more than more role i.e., a light switch can both sense its state, and provide an action, turn on or off.


\section{Typical Use Cases} % (fold)
\label{sec:typical_use_cases}
This section will demonstrate several typical use case scenarios of different combinations of devices that an ``Internet of Things'' around the home may be composed of, with some necessary requirements in order to operate correctly.

\begin{itemize}
	\item Logging % single devices, no complex relationship
	\begin{itemize}
		\item The logging application runs on a controller device interacting with one or more sensors. The controller requests data from a temperature sensor and logs it to a file locally. The purpose of such an application can be to help with understanding the state of an area, room or home over a larger period of time, from which some assessment can be made i.e., one room is always warmer than the other during the afternoon. 
	\end{itemize}
	\item Presence detection lighting % Multiple devices, complex relationship
	\begin{itemize}
		\item The presence detection application runs on a controller device interacting with both a sensor and actuator devices. Within certain areas of the environment motion detectors are placed to detect the presence of a person in a particular space. This data is then communicated back to the controller and used to determine which lights to command to turn on or off. The purpose of this application would be to reduce the energy costs associated with lights unnecessarily left on.
	\end{itemize}
	\item Wash cycle complete
	\begin{itemize}
		\item The wash cycle complete application runs on a controller interacting with a sensor and an actuator. The actuator could be any form of communication medium that can alert the user i.e., sending a text message or tweet. This application would enable a washing machine to act as a sensor, relaying information back to the user(time left) and when a wash cycle has completed or requires further attention the user could be alerted in some form so that they can attend to it.
	\end{itemize}
	\item Direct control and observation
	\begin{itemize}
		\item A direct control application would allow a user to interact with the ``Internet of Things'' directly through a web browser or smartphone application. Such an application would operate on a controller device and allow a user to dynamically create new rules and relationships between existing devices as well as directly command devices such as lights, thermostats etc.
		This would allow a user to easily create and destroy relationships between devices without reconfiguring or restarting the network of devices.
	\end{itemize}
	\item Multi-decision heating
	\begin{itemize}
		\item A multi-decision application, such as a heating system, would take advantage of an arrangement of multiple different types of sensors and existing services. These sensors could include: temperature sensors to sense the current temperature of different spaces, presence sensors to detect where a person in the environment, location sensors to detect if someone is actually home. The combination of these sensors could allow the system to detect whether the environment should be heated, and if so when and by how much. Building upon the initial use case of logging, a logger could be used to create a basic schedule of when people are in the environment so that heating schedules can be created, this would allow the environment to be heated upon the arrival of someone.
	\end{itemize}
\end{itemize}

% subsection typical_use_cases (end)

\section{Functional Requirements} % (fold)
\label{sec:functional_requirements}
This section describes the functional requirements which were gathered and are based upon the needs of the use cases discussed in the previous section.
% section functional_requirements (end)
\begin{enumerate}
	\item Support 3 device roles; Sensor, Actuator, Controller
	\begin{itemize}
		\item These 3 roles of devices are needed to form the basic closed loop interaction model necessary for the ``Internet of Things''.
	\end{itemize}
	\item Configuration of device attributes both at compile time and run time i.e., name, type, frequency, etc.
	\begin{itemize}
		\item This is necessary in order make each device distinct from another, as well as customise it based on its needs, either in terms of performance, power constraints or otherwise. By enabling run time configuration it simplifies updating the devices on the network without the need to recompile.
	\end{itemize}
	\item Discovery/querying of devices
	\begin{itemize}
		\item Discovery of other devices on the network is a core feature of a distributed ``Internet of Things'' network, without this the network can't dynamically grow by finding new devices. Querying of devices is to determine compatibility of devices before making a connection, by filtering out requests which don't match a devices attributes network congestion can be reduced.
	\end{itemize}
	\item Negotiation and connection to devices
	\begin{itemize}
		\item Negotiation is necessary to determine the characteristics of the relationship between two devices, which is based on the demands of the device initiating the connection and whether or not the recipient can or wants to meet that demand i.e., a device can request a certain frequency of readings from another device which may or may not be able to match it.
	\end{itemize}
	\item Closing connections to devices
	\begin{itemize}
		\item Devices need to be able to gracefully close any formed connections, which could be due to limited power constraints or no longer needing the resources that the connection offers. These connections also must be closed gracefully, in contrast to just not responding/dissappearing off the network, in order to ensure that a device's resources are not wasted by keeping state for a terminated connection, which could be confused for transient communication loss.
	\end{itemize}
	\item Sensors must be able to send data to controllers
	\begin{itemize}
		\item A sensor's core functionality is collecting data through sensing the environment, this data may be of interest to other devices on the network and therefore must able to to send it to them. These data transmissions also need to be sent at a set frequency, either configured or negotiated, so that a receiving device knows when to expect data.
	\end{itemize}
	\item Actuators must be able to receive commands from controllers
	\begin{itemize}
		\item An actuator's core functionality is interacting with the environment, without informed control of the device it has no use by itself. Therefore the device must be able to accept commands from other devices, such as controllers, in order to interact with the environment with a purpose.
	\end{itemize}
	\item Controllers must be able to send commands to actuators and receive data from sensors
	\begin{itemize}
		\item A controller's purpose in the network is to form connections to other devices to create the closed loop system, which in order to do this, must be able to receive data from sensors and send commands to actuators. 
	\end{itemize}
	\item All device roles must support connections to multiple devices 
	\begin{itemize}
		\item This is necessary in order to allow a device to form relationships to several other devices, which not makes better use of the available resources within each device but also allows for devices to form multiple, more complex relationships between different devices i.e., a temperature sensor connecting to one controller which logs its data and to another controller which is connected to an actuator that controls the thermostat.
	\end{itemize}
	\item Liveness checks through pings
	\begin{itemize}
		\item Due to the distributed and unreliable nature of the network environment, devices need to be able to check the liveness of the connections to ensure devices are still active, and for those that aren't, react gracefully by cleaning up the expired connection and perhaps try to discover new devices.
	\end{itemize}
	\item Application Protocol Interface
	\begin{itemize}
		\item This is necessary for applications to be built on top of the device roles, which will utilise the capabilities of the underlying network of ``Things'' to create rich and powerful systems in the ``Internet of Things''.
	\end{itemize}
\end{enumerate}

\section{Non-Functional Requirements} % (fold)
\label{sec:non_functional_requirements}
This section describes the non-functional requirements based on the expectations of the system, its performance characteristics and the requirements for developing the project as a whole.
\subsection{System Requirements} % (fold)
\label{sub:system_requirements}
\begin{itemize}
	\item Closed loop system
	\begin{itemize}
		\item The system must be able to sense, process and react within the network itself, to create a fully reactive closed loop system.
	\end{itemize}
	\item Lightweight
	\begin{itemize}
		\item Because of the target systems, constrained devices, the system must be lightweight enough, both in terms of the protocol data unit and use of system resources, so that devices such as the TelosB mote can run it.
	\end{itemize}
	\item Scalable, from small to large networks of devices
	\begin{itemize}
		\item Because of the rapid adoption of technology within our environment, the system must be able to scale from small to large networks in order to support the use of 10's if not 100's of devices with a single network, without causing unnecessary congestion and still ensuring the same performance of smaller networks.
	\end{itemize}
	\item Cross platform compatibility
	\begin{itemize}
		\item Due to the heterogeneous nature of hardware and software platforms available today, the system must be designed in such a way so that it is both hardware and software independent, enabling it to be implemented on these platforms as well as across a variety of networks.
	\end{itemize}
	\item Simple to extend
	\begin{itemize}
		\item Due to the constantly evolving nature of technology the system must be extensible in order to support new types of devices, and also be able to support new payloads specific to these devices.
	\end{itemize}
	\item Reliability of essential communications
	\begin{itemize}
		\item Because of the reliability concerns regarding the underlying network, building in some reliability is necessary to ensure that critical communications are successful, such as forming connections between devices. 
	\end{itemize}
	\item Basic assurance of correct data
	\begin{itemize}
			\item Due to the reliance upon data within the network in order to sense, process and react to the environment, all data transferred must be reliable, therefore the need for some checks to ensure it is correct when it reaches the destination are needed. 
	\end{itemize}
	\item Network resilience to device failures
	\begin{itemize}
		\item Because of the unpredictable nature of the environment and the limited capabilities of some devices (battery powered), the system must react gracefully to device failures, without the entire system being rendered in-operational.
	\end{itemize}
	\item Liveness monitoring
	\begin{itemize}
		\item The system must be able to monitor and react to the liveness of devices in the network i.e., if a device becomes unreachable, close connection gracefully and clean up.
	\end{itemize}
	\item System performance
	\begin{itemize}
		\item Because of the real-time nature of the ``Internet of Things'', the system must perform suitably, so that data is received and commands are carried out within acceptable real-time bounds i.e., receiving temperature readings for an hour ago are of no use to a device which shows the user the current temperature.
	\end{itemize}
	\item Device resources
	\begin{itemize}
		\item The system must consume as little device resources as possible in order to support the development of user applications on top.
	\end{itemize}
\end{itemize}
%How the system should perform
%LIGHTWEIGHT, SCALABLE
%Resilience, discovery, reliability, pinging

% subsection system_requirements (end)

\subsection{Development Constraints} % (fold)
\label{sub:development_requirements}
\begin{itemize}
	\item The system must be researched, designed and developed within a limited time frame, about 20 weeks.
	\item The choice of platforms for implementing the design is based on the availability of constrained devices between the School of Computing Science and the author.
	\item The choice of operating system for the constrained device is based on prior experience with similar systems and platform languages available to the OS. The Contiki OS uses a minor variation of the C programming language, making it quicker and simpler to get started based on the author's experience with C, as well as simpler to keep the system (mostly) platform independent.
\end{itemize}
% subsection development_requirements (end)
\subsection{Development Assumption} % (fold)
\label{sub:development_assumption}
In order to design and implement the proposed system, the scope of the problem needs to the limited, therefore it becomes necessary to define several assumptions.
\begin{itemize}
	\item The number of possible devices types is limited, no more than 255.
	\item Sensor data is not critically important, so packet loss is acceptable.
	\item The system will be used in an environment where security is a non-issue (but MUST be considered as future work before deployment).
\end{itemize}
% subsection development_assumption (end)

% section non_functional_requirements (end)




%Heavy weight Java
%Centralised
% selective reliability, ephemeral data
%XAP wasteful, doesn't scale
%not easy to parse
%Some request response
%Simple to implement, simple to extend, MULTICAST, UNICAST, 
%Joining devices together rather than lone wolves connecting, cheapening links


