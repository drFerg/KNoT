% Requirements of the hardware independent design
 
% Looking at xAP, JMS, what's wrong, how can that be resolved
% what are typical use cases/ types of devices

As discussed in the previous chapter, many different platforms and protocols exist to create an ``Internet of Things'' and most take very different approaches. This chapter will discuss the requirements for developing a new system better suited for constrained devices and the environment of the ``Internet of Things'', as well as highlight several drawbacks of some the previously mentioned systems and explain how such problems can be resolved with this new approach.

\section{Primary Requirements}
\label{sec:primary_requirements}
Due to the nature of the target hardware platforms, constrained devices such as Arduino and TelosB, heavyweight approaches become infeasible i.e., Java \& JMS. Therefore an extremely lightweight and scalable protocol, both in terms of the protocol data unit (pdu) and the complexity in processing and managing the runtime, is a primary requirement.

Secondary to this, because of the often unstructured and ad-hoc design of a typical ``Internet of Things'' network, building a centralised system i.e, a name server for device search, is an unnatural fit with the network. A single failure in the server could bring the entire network down, whilst all other devices are fully operational and no way to communicate with each other.
Instead, creating a distributed system which gives all devices the power and control to communicate with each other, relieves the network of a single point of failure and can help spread the load across the network. % ADD SOMETHING TO DO WITH DISCOVERY

Taking into consideration the typical types of devices connected to the ``Internet of Things'', three general classifications of devices can be discerned; a sensor, an actuator and a controller. Sensors, anything from a temperature sensor to an open door sensor, provide a continuous service to other devices in the form of real-time data, usually from the outside world. Similarly, actuators also provide a service to other devices in the form of an interaction with the outside world, such as a speaker, light switch or thermostat. Lastly, controllers are devices which orchestrate the ``Things'' in the network, forming relationships between devices and creating useful interactions between the digital and real world i.e., connecting to both a door sensor and a light switch, when the door opens the light turns on. Whilst there are these distinct classifications, it is often the case that one class of device could overlap with another i.e., a light switch can both sense, its state, and provide an action, turn on or off.


\section{Typical Use Cases} % (fold)
\label{sec:typical_use_cases}
This section will demonstrate several typical use case scenarios of different combinations of devices that an ``Internet of Things'' around the home may be composed of, with some necessary requirements in order to operate correctly.

\begin{itemize}
	\item Logging % single devices, no complex relationship
	\begin{itemize}
		\item The logging application runs on a controller device interacting with one or more sensors. The controller requests data from a temperature sensor and logs it to a file locally. The purpose of such an application can be to help with understanding the state of an area, room or home over a larger period of time, from which some assessment can be made i.e., one room is always warmer than the other during the afternoon. 
	\end{itemize}
	\item Presence detection lighting % Multiple devices, complex relationship
	\begin{itemize}
		\item The presence detection application runs on a controller device interacting with both a sensor and actuator devices. Within certain areas of the environment motion detectors are placed to detect the presence of a person in a particular space. This data is then communicated back to the controller and used to determine which lights to command to turn on or off. The purpose of this application would be to reduce the energy costs associated with lights unnecessarily left on.
	\end{itemize}
	\item Wash cycle complete
	\begin{itemize}
		\item The wash cycle complete application runs on a controller interacting with a sensor and an actuator. The actuator could be any form of communication medium that can alert the user i.e., sending a text message or tweet. This application would enable a washing machine to act as a sensor, relaying information back to the user(time left) and when a wash cycle has completed or requires further attention the user could be alerted in some form so that they can attend to it.
	\end{itemize}
	\item Direct control and observation
	\begin{itemize}
		\item A direct control application would allow a user to interact with the ``Internet of Things'' directly through a web browser or smartphone application. Such an application would operate on a controller device and allow a user to dynamically create new rules and relationships between existing devices as well as directly command devices such as lights, thermostats etc.
		This would allow a user to easily create and destroy relationships between devices without reconfiguring or restarting the network of devices.
	\end{itemize}
	\item Multi-decision heating
	\begin{itemize}
		\item A multi-decision application, such as a heating system, would take advantage of an arrangement of multiple different types of sensors and existing services. These sensors could include: temperature sensors to sense the current temperature of different spaces, presence sensors to detect where a person in the environment, location sensors to detect if someone is actually home. The combination of these sensors could allow the system to detect whether the environment should be heated, and if so when and by how much. Building upon the initial use case of logging, a logger could be used to create a basic schedule of when people are in the environment so that heating schedules can be created, this would allow the environment to be heated upon the arrival of someone.
	\end{itemize}
\end{itemize}

\section{Functional Requirements} % (fold)
\label{sec:functional_requirements}

% section functional_requirements (end)
\begin{enumerate}
	\item Support 3 classifications of devices
	\begin{itemize}
		\item This includes; Sensors to sense the environment, Actuators to interact with the environment and Controllers to orchestrate the network of devices.
	\end{itemize}
	\item Configuration of device attributes, such as name, type, frequency, etc.
	\begin{itemize}
		\item Devices in the network should be configurable programmitcally 
	\end{itemize}
	\item Discovery/Querying of devices, use of multicast
	\item Negotiation and Connection to devices
	\item Closing Connections to devices
	\item Sending data from Sensors to Controllers
	\item Sending commands from Controllers to actuators
	\item Aliveness checks through pings
\end{enumerate}

\section{Non-Functional Requirements} % (fold)
\label{sec:non_functional_requirements}

\subsection{System Requirements} % (fold)
\label{sub:system_requirements}
\begin{itemize}
	\item Lightweight
	\item Scalable both in device power and quantity
	\item Basic reliability of data due to unpredictable network conditions
	\item Network resilience to device failures
	\item Network reliability for essential communications
	\item Aliveness monitoring
	\item System performance
\end{itemize}
%How the system should perform
%LIGHTWEIGHT, SCALABLE
%Resilience, discovery, reliability, pinging

% subsection system_requirements (end)

\subsection{Development Requirements} % (fold)
\label{sub:development_requirements}
\begin{itemize}
	\item The system must be designed and developed within a limited time frame, about 20 weeks.
	\item Choice of devices is based on the availability of constrained devices between the School of Computing and the author, therefore the use of TelosB motes is imperative.
	\item The choice of operating system for the constrained device is based on prior experience with similar systems and platform languages available to the OS. The Contiki OS uses a small variant of C, making it quicker and simpler to get started based on the author's experience with C as well as simpler to keep (mostly)platform independent.
\end{itemize}
% subsection development_requirements (end)

% section non_functional_requirements (end)



% subsection typical_use_cases (end)
\begin{comment}

\end{comment}


%Heavy weight Java
%Centralised
% selective reliability, ephemeral data
%XAP wasteful, doesn't scale
%not easy to parse
%Some request response
%Simple to implement, simple to extend, MULTICAST, UNICAST, 
%Joining devices together rather than lone wolves connecting, cheapening links








% section problems_encountered (end)